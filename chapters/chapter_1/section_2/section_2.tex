\documentclass[../../../Meccanica_quantistica]{subfiles}


\begin{document}

\section{Sovrapposizione e indeterminazione}
  \label{sec:sovrapposizione-e-indeterminazione}

  L'introduzione della funzione d'onda e il conseguente
  abbandono del determinismo comporta una innegabile
  complicazione nella teoria. \\
  Questa è però \say{aggirabile} grazie al
  \textbf{
    \textit{principio generale di sovrapposizione degli stati}
  }. \\  
  Per comprenderlo è però necessario comprendere il concetto
  stesso di \textbf{stato}.

  % Il concetto di "stato"
  \subfile{subsections/subsection_1}

  % Il principio generale di sovrapposizione degli stati
  \subfile{subsections/subsection_2}

  % Il significato di uno stato come sovrapposizione di altri due
  \subfile{subsections/subsection_3} 

  % Formulazione matematica del principio di sovrapposizione
  \subfile{subsections/subsection_4}

\end{document}



















