\documentclass[../../../../Meccanica_quantistica]{subfiles}

\begin{document}
\subsection{Il principio generale di sovrapposizione degli stati}
    \label{
      subsec:il-principio-generale-di-sovrapposizione-degli-stati
    }
    Il principio generale di sovrapposizione degli stati — valido
    per entrambe le definizioni appena date di stato — consiste
    nell'ipotesi che ogni stato sia esprimibile come
    \textit{sovrapposizione} di altri due o più stati in un numero
    infinito di modi (moralmente si sta parlando di combinaizoni
    lineari, lo vedremo meglio quando introdurremo il formalismo
    matematico) e al contempo che un numero arbitrario di modi
    possano essere sovrapposti creando un nuovo stato. \\
    N.B.: sovrapporre gli stati può risultare più o meno utile in
    base al problema preso in esame.

\end{document}