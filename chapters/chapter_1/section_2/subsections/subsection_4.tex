\documentclass[../../../../Meccanica_quantistica]{subfiles}

\begin{document}

\subsection{
  Formulazione matematica del principio di sovrapposizione
}
  \label{
    subsec:formulazione-matematica-del-principio-di-sovrapposizione
  }
  Il processo di sovrapposizione risulta come una specie di
  processo additivo il cui risultato è dello stesso tipo degli
  addendi.
  Gli oggetti matematici più indicati sono quindi i vettori
  (qualcuno potrebbe affermare che anche un gruppo ha questa
  proprietà, ma hey noi non siamo Dirac). \\
  Per trattare la meccanica quantistica utilizzeremo quindi
  \textbf{vettori a componenti complesse}, ma avremo bisogno di
  infinite dimensioni (per il momento la cosa cade dal cielo). \\
  I vettori associati ad uno stato del sistema sono denominati
  \say{\textit{[vettori] ket}} e il ket $A$ è indicato come
  $\ket{A}$. \\
  Essendo per l'appunto vettori, ogni combinazione lineare di ket è
  a sua volta un ket, ma supponendo una dipendenza da una variabile
  $x$ vale anche
  \begin{equation*}
    \int \ket{x} dx = \ket{Q}
  \end{equation*}
  Un ket ottenibile come combinazione lineare di altri ket è detto
  dipendente da essi, e un insieme di ket si dice indipendente se
  nessuno dei suoi elementi è esprimibile come c.l. degli altri. \\
  Fatte queste premesse è ora possibile sfruttare l'equivalenza tra
  il concetto di sovrapposizione di stati e quello di combinazione
  lineare di ket, dalla quale è banale notare che:
  \begin{equation*}
    c_1 \ket{A} + c_2\ket{B} = \ket{C} \Rightarrow
    \frac{c_1}{c_2} \ket{A} - \frac{1}{c_2} \ket{C} = \ket{B}
  \end{equation*}
  E giungere quindi alla conclusione che gli stati sovrapposti per
  creare altri stati sono a loro volta esprimibili in funzione degli
  stati generati e dei restanti stati sovrapposti. \\

  L'intuito ci dice però che sovrapponendo uno stato con se stesso
  io non debba fare altro che ritrovare lo stato stesso,
  matematicamente, sfruttando le proprietà dei vettori:
  \begin{equation*}
    c_1 \ket{A} + c_2 \ket{A} = (c_1 + c_2) \ket{A}
  \end{equation*}
  E giungiamo necessariamente alla conclusione che il ket $c
  \ket{A}$ sia associato allo stesso stato del ket $\ket{A}$.
  Ad essere associato allo stato non è quindi il ket in sé ma la usa
  \say{direzione}, se di direzione si può parlare \textit{(N.B. nel
  caso in cui $c_1 + c_2 = 0$ il ket risultante è chiaramente
  nullo, ma questo è solo un caso particolare che non trattiamo ora,
  anticipiamo però che il ket nullo non corrisponde a nessuno
  stato)}. \\
  La dipendenza dalla sola direzione ha anche un'altra importante
  implicazione: dato che lo stato risultante non varia a seguito di
  una moltiplicazione per fattore complesso, a determinare il
  risultato non sono le costanti moltiplicative dei ket che sommo,
  ma il loro rapporto, dato che esso rimane invariato anche dopo la
  moltiplicazione
  ($\frac{c_1}{c_2} = \frac{d \cdot c_1}{d \cdot c_2}$).
  Ciò chiaramente riduce il numero di gradi di libertà del nostro
  risultato da 4 a 2: 4 perché ogni numero complesso è definito da
  parte reale e parte immaginaria, quindi 2 fattori moltiplicativi
  $\to$ 4 parametri reali, ed ora 2 perché da 2 fattori
  complessi si passa nel pratico ad 1: il loro rapporto.



\end{document}