\documentclass[../../../../Meccanica_quantistica]{subfiles}

\begin{document}

\subsection{
    Il significato di uno stato come sovrapposizione di altri due
  }
    \label{
      subsec:Il-significato-di-uno-stato-come-sovrapposizione-di-altri-due
    }
    Per comodità tratteremo il caso di uno stato originato dalla
    sovrapposizione di soli due stati, tuttavia l'idea di come
    generalizzare le nozioni qui introdotte ad un numero superiore
    dovrewbbe essere abbastanza semplice (soprattutto rifacendosi
    all'idea delle combinazioni lineari).

    Per sovrapporre due stati è necessario assegnare sia un peso a
    ciascuno di essi sia una differenza di fase alla coppia (il
    significato di questa fase ci sarà dato dalla teoria
    matematica). 
    
    L'esempio della sovrapposizione dei due stati $A \rightarrow a$
    e $B \rightarrow b$ mette in luce la corretta interpretazione
    del processo di sovrapposizione: \textbf{uno sato ottenuto dalla
    sovrapposizione di altri due ha comportamenti intermedi ad essi
    non perché presenti caratteristiche intermedie bensì poiché si
    comporta talvolta come l'uno e talvolta come l'altro.}

\end{document}