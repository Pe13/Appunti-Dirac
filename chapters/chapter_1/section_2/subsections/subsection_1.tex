\documentclass[../../../../Meccanica_quantistica]{subfiles}

\begin{document}

\subsection{Il concetto di "stato"}
    \label{subsec:il-concetto-di-stato}
    Dato un insieme di particelle e un set di forze/vincoli, sono
    chiaramente possibili un continuo di moti parametrizzato dalle
    condizioni iniziali sel sistema. \\
    Ognuno di questi possibili moti rappresenta uno stato. \\
    Dovendo occuparci di meccanica quantistica è impensabile
    identificare queste condizioni iniziali nelle posizioni e
    velocità di ogni particella (un saluto ad Heisenberg); lo stato
    di un sistema sarà determinato dal massimo numero di condizioni
    o dati teoricamente possibili senza mutua interferenza (sempre
    Heisenberg) o contraddizione. \\
    Nel pratico con la parola stato si possono indicare sia
    l'intero moto (appena detto, e alle volte chiamato \say{stato di
    moto}), sia un istante dello stesso;
    volendo utilizziare un formalismo matematico si potrebbe dire
    che con il termine stato si indicano rispettivamente sia
    $\vec{f}(t)$ che $\vec{f}(t_0)$ ($\vec{f}$ scritto come vettore
    perché deve restituirmi tutte le variabili del moto).

\end{document}
